%%-------------Abstract----------------- 
\doublespacing 
\chapter*{Abstract}
\addcontentsline{toc}{chapter}{Abstract}

Artificial intelligence is in an ideal position to use research in other fields
to improve itself.  In this thesis, psychological and cognitive
models of learning, as well as human memory models, have been used to simulate problem
solving.  This was done by applying human learning and memory to a problem in
natural language processing, the word sense disambiguation problem. The question being 
answered is ``can human-like memory improve accuracy in a particular area of artificial
intelligence, specifically word sense disambiguation?". This involved exploring the 
various models and presenting how they could fit into a computer system.  The model was
implemented and the details for how it came together, what worked and what did not were
presented.  Finally, the system was tested, using an open source competition dataset, 
and compared with other systems using the dataset.  The results ranked the system among
the top five contestants. The results also show a potential for future accuracy 
improvements in determining word sense, as well as a chance to better model human memory 
and learning.
