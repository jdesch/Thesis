%%-----------Intro -------------%% 
\chapter{Introduction}

The pursuit of better artificial intelligence has brought an analytical eye to
numerous fields and disciplines.  While the idea of computers or automata
emulating human intelligence is as old as the greek myth of Hephaestus and his
bronze man Talos\cite{TALOS}, the difficulty of the problem has prevented the 
realization of the true ``Artificial Man''.  Research into math, statistics, logic, and general
computer science has improved the overall effectiveness of the field. Matching
various techniques to sub-fields within artificial intelligence has further
improved specific areas and cases (e.g. formal grammars and Natural Language
Processing, symbolic logic and constraint problems, statistics and inference
techniques). While current improvements have been related to better statistical
algorithms, more descriptive mathematical models and more powerful computer
hardware, the necessity to try other techniques or reinvent old techniques has
emerged as the field matures.  Although human sentience is largely a mystery, our basic problem-solving abilities offer a great model for an intelligent framework.

When searching for a problem domain for this thesis, it was necessary to find a
problem that we (humans) are exceedingly well suited for.  While there are
numerous areas where a human outmatches a computer, the area had to have a well
defined problem and be easily testable against non-human artificial intelligence
systems.  We are supreme fuzzy pattern matchers.  Even today, no computer is as
capable at matching a name to a face as quickly as we are.  Pattern matching relates
to our language, humour, recent events, metaphors, expressions, context, and
double meanings. The human brain handles and works with ambiguous situations and
statements extremely well.  With this last point in mind, natural
language processing became a good candidate and the specific task of word sense disambiguation seemed
to have a well defined problem space.  As the title of the problem suggests,
word sense disambiguation relates to the action of understanding which meaning
an ambiguous word has in the context of a particular sentence. Current
artificial intelligence systems perform well below the required effectiveness to
make adding this to existing technologies useful.

The motivation for this thesis is two-fold.  The main focus is to adapt human
intelligence and the human learning process, specifically human memory models,
to an artificial intelligence system. This would involve looking at various models
and taking what is useful or fits well with other systems and combining them to 
promote successful problem solving.  The second focus is to show that this
adapted system can be useful in solving hard artificial intelligence problems,
specifically word sense disambiguation.  Recently, discussion in some of the
academic journals put forth the idea of using the latest ideas in biology, 
social science,
psychology, and cognitive science to improve aspects of artificial intelligence.
Looking into the background of the word sense disambiguation problem, there was
no indication of work that used even a basic biological model. With this gap,
and the prospective good match of the human intelligence framework or the human
memory model to the word sense disambiguation problem, it seemed like a prospective
avenue to further the field.  Work done in this area can open up a few interesting
avenues. First, if it proves interesting or useful in the general case, it
promotes another method for improving current artificial intelligence systems.
Second, if it proves useful specific to the word sense disambiguation problem it
could help improve the field and help other areas within natural language
processing.  Not only that, even with a partial success it would be useful to
identify where in the field of artificial intelligence the human models could
make an impact, and to apply it.  Finally, if little use comes out of it, or the
implementation of the human memory model is flawed, a researcher may gain
inspiration from the ideas presented.

Artificial intelligence research has shown that statistical methods perform well
in general; however, as the data sets grow and as the problem becomes more general, 
often correct answers and small patterns get lost in the noise.  A number of techniques have been developed
to combat this problem; when these techniques are combined, small incremental
gains in accuracy are achieved \cite{ENSEMBLE2}.  Modelling human memory may
provide insight into improving accuracy when working with both large data sets
and loosely coupled data areas.  Humans using general intelligence, past
memories and other conscious and non-conscious mental abilities are able to
instinctually discover minute patterns in large sets of data, quickly
\cite{SUBQUESTION}.  Computer algorithms attempt to statistically determine patterns;
however, the ability to repress certain patterns, while weighing other patterns
more heavily, is more effectively done by a human.  This ability allows for
humans to cut out a vast majority of the data set through the use of logic,
experiences and understanding.  These qualities certainly require more than just
memory; however, without memories, none of these abilities could occur.

It is hypothesized that the incorporation of human-like memory should provide
improvements in the area of word-sense disambiguation. The problem requires
general knowledge and the ability to recall information with a bias dependent on
the context of each paragraph, sentence and word.  Developing human-like memory
in concert with current statistical techniques should provide more accuracy in
word sensing and provide insight into the impact of imperfect memory on
difficult, general problems.

The question then is: can human-like memory improve accuracy in a specific area
of artificial intelligence, specifically word sense disambiguation?

Chapter 2 will present some background information on artificial intelligence,
natural language processing (specifically the word sense disambiguation
problem), and early artificial intelligence work done in biological systems. It
will also give an overview of some relatively modern research in human memory models
that will be adapted and used throughout the thesis.  Chapter 3 discusses the
methodology of the theoretical models chosen. This chapter illustrates the
``why'' of the models, it highlights what could be potentially useful in its
implementation and gives insight into design considerations.  Also presented are
what areas are lacking or missing and what has been done to fulfill the
requirement, while outlining what is out of scope or too difficult to implement.
Chapter 4 presents the implementation details, or the ``how''.  This chapter
shows the details of the test implementation, the tools used, the data set, and
the algorithms used to implement the chosen theoretical models.  Chapter 5
presents and discusses the quantitative and qualitative results. Finally,
Chapter 6, the conclusion, discusses the overall success of the work presented and
how it relates to the two-fold motivation and goals mentioned earlier in this
chapter. It also puts forth interesting avenues for future areas of research on
the topic.
