\chapter{Background} 

\section{Natural Language Processing}

Theorists have posited that true intelligence would allow computers to carry 
on insightful conversations. The conversations would be of
such high quality that the human would be unable to determine that they were in
fact communicating with a computer\cite{TURING_TEST}.  Natural language
processing(NLP) is theorized to be AI-Complete \cite{AICOMPLETE}, meaning that
to solve this problem would give computer scientists the insight required to
create computer programs as intelligent as humans.  Initial work in this area
showed remarkable success: in automatic translations between languages
\cite{EN_RUS}, the application to restricted words and
languages\cite{AI_MODERN} all provided good results.  However, application to
the general problem of linguistics was much slower.  Toy programs were developed
that simulated conversation; these chatterbots could be programmed to resemble
intelligent beings.  However, the programs lacked the knowledge depth and
application required to be considered truly intelligent, instead relying on
canned responses.   

NLP requires numerous approaches and techniques to
evaluate human texts. Humans often require multiple types and levels of
processing to correctly comprehend language \cite{NLPENCY}. Most of these levels
are used together to differentiate between the various meanings, as each of the
levels adds a method of disambiguation to the context of the linguistic
structure.  NLP programs often use multiple levels and combinations of
linguistic analysis; it is in this that most programs differ \cite{NLPENCY}.
The various approaches to analyzing NLP include the following. \begin{itemize}
\item The symbolic approach: This often involves the use of a symbolic rule
system developed by humans with an emphasis on known representation and language
processing algorithms.     \item The statistical approach: This uses observable
information modelled from the linguistic world based on probabilities, commonly
using Hidden Markov Models.     \item The connectionist approach: This combines
statistical approaches with various components of representation, allowing
manipulation of the underlying models and sub-problems within NLP, while
continually improving confidence in answers. \end{itemize} 

\subsection{Word Sense Disambiguation}

One of the sub-problems of natural language processing
is word sense disambiguation (WSD).  WSD is the act of identifying the meaning
of a word in a sentence when the word has multiple meanings.  An example is in
the following two sentences: ``I heard a loud pop.", ``I enjoy an occasional
pop.".  In the first sentence, the word ``pop" refers to a noise; in the second
sentence, ``pop" refers to a drink.  Humans, with the use of a large volume of
collected information and the skill set developed by working in the ambiguities
of language, are remarkably effective at sorting out the meaning of the words.
Because of the general intelligence required to solve these problems, this
problem is also classified as AI-Complete \cite{BARHILL}.   The problem was
encountered initially in automated machine translation; in fact, it was one of
the first ``hard" road blocks \cite{WSDAS}.  Initial approaches considered using
computers to automate the understanding of languages; however, at the time,
there was an insufficient collection of data \cite{WSDAS}.  In the 1990's with
the prevalence of the internet came a large collection of electronic documents.
With computational power also increasing greatly over the previous twenty years,
statistical and computationally expensive techniques were implemented and
allowed for improved accuracy.  From the Senseval/Semeval contests
\cite{SEMEVAL}, we are able to see how various researchers performed.  The
Semeval contests outlined a number of challenges over the years; these
challenges included different dictionaries and different goals as related to
word sense disambiguation.  The trends from the competitions are quite difficult
to determine; however, on the whole, performance has slowly increased from
contest to contest.  Successful techniques in WSD include memory-based learning
\cite{GAMBL, PARAM_OPT, MEM-BASED}, instance based learning
\cite{INSTANCE_LEARN}, decision lists \cite{YarrowskyDL}, ensemble methods
\cite{ENSEMBLE}, kernelization \cite{KERNEL}, as well as various knowledge
hybrid methods \cite{HAWKWSD}.  

The last forty years in artificial
intelligence have seen the rise of statistical techniques, which proved to be
useful tools for a number of hard problems. Over these years, both technology
and theories in related fields have come a long way.  It is these new theories
and tools that will be used to challenge ideas and techniques that have held
true in artificial intelligence for years.  Biological theories about the inner
workings of the brain show promise for a number of areas of computer science,
from parallel architectures to general intelligence \cite{TOBICA, ERSATZ}.  The
investigation into biologically inspired cognitive architectures opens up a
number of areas \cite{BICA}.  Just as early computer scientists looked to the
brain in creating computational and artificial intelligence models, again are a
number of AI researchers looking towards the theoretical aspects of the brain
and mind.  New theories into cognitive awareness, thought processes, learning,
memory, understanding, self-examination, along with improvements in
neuroscience, brain chemistry understanding and psychology, allow for a number
of possible research avenues. 

One area that has a rather small volume of work
in computer science, but has a remarkable importance in biology, psychology,
neuroscience and learning science, is that of human memory.  Three interesting
human memory models come from the psychology and cognitive science fields.  The
theories are briefly described in the following sections. 

\section{Human Memory Models}

\subsection{Baddeley and Hitch} 
\begin{figure}[htp]
	\begin{center}
		\includegraphics[width=0.85\linewidth]{figures/models/BH-Model}
		\caption{Baddeley and Hitch Model}     
	\end{center}     
	\label{fig:BH_MODEL}
\end{figure} 

In Baddeley and Hitch's model \cite{Work_Mem, Episodic_Buff}, three
systems (the phonological loop, the visuo-spatial sketchpad and the episodic
buffer) are responsible for short-term maintenance of their respective
information, whereas the fourth system, the central executive, is responsible
for information integration and system coordination.  The central executive
system is responsible for directing attention to relevant information,
suppressing irrelevant information and unnecessary actions, and providing the
ability to multitask. Furthermore, the central executive also provides a
supervisory system, which can guide the other components of working memory back
into a stable working environment. The phonological loop takes auditory verbal
information and written language and encodes them into an auditory code. This
information is then repeatedly cycled in the temporal order to refresh
contextual information and prevent decay. The visuo-spatial sketchpad consists
of the ``visual cache" and the ``inner scribe''.  The visual cache stores
information relating to form and colour, where the inner scribe stores
information relating to movement, speed and distances. The working memory allows
manipulation of this environment to enable planning and spatial understanding.
The episodic buffer provides a temporary memory component, which integrates the
loop systems and other acquired information, producing a ``unitary episodic
representation''\cite{Episodic_Buff}.  Through the chronological sequencing the
applications of both loops are more easily transported to long-term memory.
Furthermore, in uniting the visual and sound loops with chronological ordering,
previously unknowable semantic information may present itself.

\subsection{Cowan} 
\begin{figure}[htp]     
	\begin{center}
		\includegraphics[width=0.85\linewidth]{figures/models/Cowan-Model}
		\caption{Cowan Model}     
	\end{center}     
	\label{fig:COWAN_MODEL} 
\end{figure}

In Cowan's theory \cite{ACTIVATION, ATTN_MEM}, working memory is part of long-term memory.  The representation used by working memory is a subset of memory
elements stored in long-term memory.  Working memory has two levels. The first
consists of activated long-term memory elements of which there can be an endless
number, as long-term memory is limitless.  The second level is the focus of
attention. This focus has a limited capacity and holds a more detailed component
on which work can be done; this level is able to contain about 4 ``chunks'' of
information.  Other cognitive scientists have added a new level to Cowan's
original theory, which has increased focus and is only able to contain a single
element. The focus aspect of Cowan's theory has a few properties. These include:
\begin{itemize}
	\item Focus is controlled by both voluntary (conscious) and involuntary 
	(unconscious) methods.
	\item Focus is required to perform many day to day tasks such as: planning, 
	understanding, and cause and effect.
	\item The level of control of focus and capacity of focus differ on an 
	individual basis.  
\end{itemize}    

This model readily lends itself to solving
problems in both a high-level idea-oriented situation and a low-level detail-oriented situation, with the ability to change between the two.  Working on and
modifying long-term memory ``chunks'' in working memory allows for a useful
instance-based memory component that can be used to focus on solving problems
and understanding.  Each application of problem solving then modifies the long-term memory instance it had used.

\subsection{Ericsson and Kintsch} 

\begin{figure}[htp]     
	\begin{center}
		\includegraphics[width=0.85\linewidth]{figures/models/Ericsson-Kintsch}
		\caption{Ericsson and Kintsch Model}     
	\end{center}     
	\label{fig:EK_MODEL}
\end{figure} 

In this theory, Ericsson and Kintsch \cite{LTM} hypothesized that
humans use specialized memory for most tasks.  They also suggest we store most
of what we read in long-term memory, linking it through retrieval structures.
Concepts can be held in working memory and can act as cues that link to the
retrieval structures and return the memories associated.  Different types of
retrieval structures include generic, domain knowledge, and episodic text.
Generic retrieval occurs deliberately, domain knowledge is retrieved through
patterns and schemas, and episodic text is retrieved through text comprehension.
Specialized memory is based on using specific knowledge of the area of memory
and applying it to the storage and encoding of information in long-term memory.
The cues act essentially as a ``hash" of the long-term memory connected to it;
this allows great memory performance by allowing rapid and flexible long-term
memory access \cite{LTM}.  Although the exact mechanism for cue creation is
unknown, the retrieval cue must be encoded to handle temporal information.
Newer information must have more of a precedent than older information, allowing
memories and their structure to change as new information is presented.  Encoded
information is not only linked by a retrieval cue, but also has semantic links
connecting the larger pieces of information to one another.

\subsection{Additional Models} 

In addition to the three well known theories
mentioned above, one area of cognitive science that may benefit memory research,
but is notoriously difficult to model and understand, is that of the unconscious
mind. The importance of the unconscious mind can be summed up by a question
asked by Greenwald, Klinger and Schuh \cite{SUBQUESTION}: ``How good is the mind
at extracting meaning from stimuli of which one is not consciously aware?'' with
answers ranging from: it is simplistic and dumb \cite{DUMBUNCONSCIOUS}, to, it
is prevalent and strongly influences nearly all higher mental processes
\cite{SMARTUNCONSCIOUS}. The effect of the unconscious has in its majority
created the complex and intelligent design necessary for living things through
adaption and natural selection \cite{UnconsciousMind}.  While the extent to
which we rely on the unconscious region of our mind is debatable, the desire to
understand how this part of our mind works, and its limits, can be found in
everything from business management journals to the New York Times best-seller
list. In an experiment, Bechara et al.\cite{Bechara2} were able to show that
participants in a rigged game were able to decide upon the advantageous strategy
well before they could consciously understand why the strategy was successful.
This unconscious understanding, or implicit memory, is a relatively new area of
research, with almost all studies completed within the last twenty-five years.
Current research suggests that this memory relies on an implicit, unconscious
piece of knowledge previously learned and that the process and structure are not
completely related to explicit memory \cite{ImplicitMem}. 

There has been some work in applying cognitve models to aspects of computer science and
artificial intelligence.  The majority of these systems focus on a symbolic 
rule based interpretation of memory and are designed more from a cognitive psychology 
point of view. Two popular systems are the ACT-R and Soar architecture.

ACT-R (Adaptive Control of Thought - Rational) began in the early 1970's where it added
a rule-based, or procedural system to the basic declaritive models of the time \cite{ACT-R}.
The work done over the last forty years added a mathematical rational framework \cite{ACT-R2}, 
created specialized models to simulate brain functions \cite{ACT-R3}, and added constructs
to reflect cortical activity \cite{ACT-R3}. ACT-R's approach to memory is to have two systems:
declaritive and procedural memory, where the procedural memory essentially acts like a neural net 
subsystem.

The Soar (State, Operator and Result) architecture, a 
symbolic cognitive architecture developed from the late 1980's with a goal of cognition
and general intelligence \cite{SOAR}.
While it does not incorporate all the models listed below,
it does use working memory in conjunction with various forms of Long-Term memory and
problem space searching in its use as a production rule-based system\cite{SOAR2}. 

The Baddeley and Hitch
theory takes the importance of bio-feedback loops and applies them to long term
and working memory, adding a unit that controls overall interaction.  This
theory briefly mentions the importance of attention; however, the Cowan model
expands on what attention and focus are and their importance to our memory.
The Cowan model also attempts to explain how our conscious mind and memory
interact, whereby greater focus on an element provides more information to the
preclusion of other memories. The Ericsson and Kintsch model adapts the memory
component itself and its association with other memories. This allows for
efficient storage and retrieval of memories, as the memories can be saved in
different ways according to the way they were retrieved and extracted.  Ericsson
and Kintsch also help to explain how the mind and memories perform some tasks so
quickly, with the retrieval structures between working memory and long-term
memory and semantic links between memory components in long-term memory.
Finally, the unconscious mind handles non-directed learning and its interaction
with memory, decision making and understanding. While there is much debate in
how the unconscious mind works, decision making and problem solving do improve
when unconscious directions are considered.
