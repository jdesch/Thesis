%%-----------------Chapter 4--------------------------- 
\chapter{Implementation}

This chapter deals with the implementation details of the system and the
algorithms that make it up. This includes specifics on the structure of the
input data, the training setup and procedures, the learning and adaptation
specifics, and how the testing was performed. Where as the last chapter brought
up the design decisions and the reasons behind them, this section deals with how
the system implemented the ideas. This section also discusses the outcomes of
the various implementation choices and attempts to further justify the design
decisions.

Before jumping into the specifics of the test implementation, a quick reminder
of the problem domain is in order.  Given a sentence, a target ambiguous word in
the sentence, can a system properly disambiguate the word, providing the correct
word sense?  Therefore, we are trying to disambiguate a word with multiple word
senses in a given sentence, where disambiguation is the action of determining
which word sense is correct in the given situation.

\section{Input Data} The input data to the test was taken from the Natural
Language Tool Kit (NLTK), an open source python project for use in linguistics
and natural language processing research and development.  The NLTK is used
specifically for its symbolic and statistical natural language processing; which
is a good fit for the topic of this thesis.  The main use of the NLTK was to
provide the Senseval-2 contest dataset. The dataset is made up of manually
tagged sentences made publicly available through the Senseval/Semeval workshops,
a workshop put on by the Association for Computational Linguistics which
promotes the semantic analysis of text. These sentences are tagged in a
reproducible manner, each sentence was added with the rule that human taggers
must agree on the sense of each word at least 90\% of the time; requiring at
least two taggers to determine the validity of each word.  The Senseval-2
dataset consists of 15225 sentences containing an average of 27 components.  The
sentences were organized in a strict XML format, with each record containing the
following elements:

\begin{itemize}     
	\item word- the targeted word and its abbreviated part of
speech. In the senseval-2 data set there are 4 words to be disambiguated:
``hard'', ``interest'', ``line'' and ``serve''. An example of this category is
simply: ``Hard-a'' with the ``-a'' being representative of its part of speech
(in this case an adverb)
	\item context- the sentence the targeted word is
found in. This consists of a set of tuples comprising each component of the
sentence; each tuple contains a stemmed word or component, and its part of
speech tag. For instance: \begin{verbatim}[('someone', 'NN'), ('has', 'VBZ'),
('to', 'TO'), ('stop', 'VB'), ('him', 'PRP'), ('.', '.')] \end{verbatim}
	\item senses- The word senses of the targeted word, taken from the WordNet 1.7
sense inventory . An example: ``HARD1'' which can be cross-referenced for its
definition 
\end{itemize}

In addition to the dataset, the NLTK also provided a mechanism for stemming and
lemmatizing each word, as well as providing a list of commonly used stop words.
This allows each sentence to be more compact, and concise, which greatly reduced
the overall complexity of the graph-based memory component.

\section{Setup}

The data was setup using five-fold cross validation in which four partitions of
the dataset were used for training, and the fifth was used for testing. Which
was then repeated five times, allowing each of the data sets to be the test
data, and having every combination of use for the training data.  This method
differed from the method used at the Senseval-2 competition, but allows for an
acceptable comparison.  The Senseval-2 competition gave the researchers a full
training set, but held back on releasing to them the data they would be testing.

\section{Training}

The training component of the system provided a non-trivial challenge.  To
create and synthesize the learning process in humans a two phase approach to
training was required.  The first phase consisted of the initial setup of both
the statistical component and the connective component. The connective component
required its ``memory connections'' to be added.  This was the process of adding
weighted edges between words that were found in the same sentence, where the
edge weight was directly proportional to the frequency of occurrences between
the two words.  The statistical component collected predefined information which
was to be used later for decision making.  This first phase ran over the entire
training data.

Once this initial processing was done, a second pass of the training material
was required.  This pass emphasized the learning process.  In this phase the
graph component setup in the previous phase was further adapted and trained to
the dataset.  The statistical component was tested for accuracy and the meta-
learning component began its initial setup and data collection on information
from the accuracy of the two other components. This included having the meta-
learning component query the statistical component to learn which words in the
sentences were the best to disambiguate on. This process attempted to eliminate
noisy, or unimportant words and select only those words with the best chance to
properly disambiguate the target word.  Selecting the best words was done by
using an algorithm which made use of the following three factors:

\begin{itemize}
	\item Completion rate: this was determined by how likely the
word was to complete successfully. Which could also be considered the
relationship between the word and the successful selection of some word sense in
both the statistical and connective algorithms.      
	\item Accuracy: the probability the word would be accurately disambiguated.     
	\item Rarity: the occurrence of a word 
\end{itemize}

Finally, once the best words were known, the number of words to choose per
sentence was then determined by the meta-learning system.

Another important aspect of the training phase is the reporting method.  This is
done through the use of a Learning Environment Data Structure (LEDS).  This data
structure carries the information used to solve the problem, collects the
statistics and computes the meta-data. This data structure is passed into both
the statistical and connective guessing process, it collects their choices and
other data required to be updated after the learning instance is completed.  It
is then passed into the meta-learning phase, in this phase it is used to help
compute the final answer. Finally, after the algorithm has made its guess, it
captures the correct disambiguation, and is used to update the long-term memory
data structure with the changes noted in the learning instance phase.

\subsection{graph-based adaptation}

The method in which graph-based adaptation took place was through the use of
collected meta-data, and instance based feed-back.  With the initial graph
setup, each word in each sentence were interconnected. Over the training data
this formed a very dense graph, with the value for each edge allowing for the
most likely location for a given word sense to be found.  The idea behind the
algorithm was each word selected as important in a sentence would vote on which
disambiguation was most likely, the word sense with the greatest number of votes
would be determined as correct.  This technique not only determined the ``best''
possibility, but gave an approximate confidence value in the chosen word. The
search performed went through each of the selected starting words, and would
look through its neighbours for the given word senses in decreasing edge weight
order. If an occurrence was found, it was selected as its vote, otherwise it
would pick the highest rated edge and perform the same search on its neighbours.
If after a meta-defined number of neighbours were checked, and no
disambiguations were found, the selected word would not vote.  If in returning
from the search there was a tie, the vote would be decided arbitrarily.
Throughout this search the path each word took and the disambiguation it voted
for was tracked and kept to be used to adapt the graph data structure.

The graph adaptation algorithm was performed after the instance problem was
completed and the answer was known.  During this phase a few outcomes were
possible: \begin{itemize}      \item the graph component correctly determined
the word and each word completed its search.  In this case each edge along the
path each selected word took had its weight increased by a meta-determined
amount.      \item the graph component incorrectly determined the word and each
word completed its search. In this case each edge along the path that led to the
wrong outcome would have its edge weight value decreased by a meta-determined
amount. This amount was often significantly larger then the increase value, in
an attempt to offset initial training imbalances. The remaining words, if any,
would not have their values adjusted.     \item the graph component correctly
guessed the word, but the majority did not finish their search. In the case
where confidence is low, the graph component would not submit a vote, but would
keep its best guessed value to help determine feedback. In this case each edge
that completed with the correct value had the value of the edge weights along
its path increased (usually by quite a bit more then a normal correct case). and
the edges that did not finish would have the edge weights along their edge
decreased by a significant meta-determined amount.     \item the graph component
guessed incorrectly and the majority did not finish their search. In this case
each of the values of the edge weights along all of the paths would be decreased
by a meta-determined amount, which in most cases was usually the largest
reduction. \end{itemize}

After completing the second run of the training dataset the words were known and
the edge-weights of the graph had been trained enough for use.  Although it has
not been confirmed if additional runs of the graph adaptation phase would be
beneficially, the length of time this phase takes, and the increased likelihood
of overtraining were the deciding factors in using one run.

\subsection{statistical adaptation}

In keeping with the unconscious mind, the statistical method was designed
specifically to be a quick, relatively simple calculation with decent results.
The best method for this, and one that seemed directly related to the actual
process was Bayesian statistics.  After each learning instance the statistical
information contained there in was added to the previously collected
information.  This information was immediately available for use in calculating
probabilities for the next learning instance.  The first part of the learning
phase, the statistical component collected the number of times words were seen
together, enumerating each sentence and each word in the sentence to its proper
disambiguation, and the disambiguation to the various words seen.  The second
phase, the optimization component, the statistical component attempted to select
the correct disambiguation based on prior knowledge, and would again add the
information to its statistical collection.  The selection algorithm initially
used a wider range of variables to determine which word sense was most likely,
this however proved to be a poor choice, as the standard selection algorithm
(most frequently selected) worked much better.  This led to the second
iteration, the use of the most frequently selected algorithm.  This algorithm
uses a frequency table to keep track of all of the different word senses of a
single word, when asked to disambiguate a given word, the algorithm returned the
disambiguation which has the highest occurrences.  As mentioned above, the
statistical component is also used to select which words are the best words to
use in a sentence to determine the disambiguation. This information is collected
in both phases, but only used in the second phase.

\subsection{meta-data collection and adaptation}

During the optimization component of the learning phase, the meta-data
collection and adaptation is completed.  This phase consists of two important
parts: \begin{itemize}     \item answer biasing- This phase adds a bias to using
the answer provided by a component in accordance with how accurate the component
is.  This uses information collected through the meta-data analysis of all
previously answered learning instances to give weights to each of the answering
components equivalent to their accuracy.     \item adaptive processing- This
component monitors and tests the pre-defined meta-values.  At each phase, a
meta-value is randomly selected from the list of all meta-values, it is
incremented based on a pre-defined ``step'' value and then is run for a number
of instances.  If after the test run the changed value being tested performs
better than average, it uses this new value as a baseline, if it performs
exceptionally it attempts to retest the value with an even larger value.  If the
changed value performs worse it reverts the value, if it performs exceptionally
worse, it attempts to retest the value with a negative ``step'', a smaller
value. \end{itemize}

The meta-variables these tests were performed on consists of two groups. The
first group being the ``desirably controlled quality'' group.  This group
consists of values not directly related to accuracy, but likely have an effect
on the results.  The main reason these were added was to have the system
optimize itself throughout the learning phase and into the more efficient phase.

desirably controlled quality \begin{itemize}     \item important words - the
number of words selected from a sentence used to determine the word sense. This
one was selected in hopes that as the algorithm became more efficient, or during
the initial learning phase, the number of words required could shrink or grow.
This would allow the algorithm to determine when it should use more words and
run slower, while maintaining or increasing accuracy; Or when it should use less
words and run faster without effecting accuracy.     \item iterations between
considerations - This represents the number of test instances between the meta-
data phases. It was expected that this value would be the last value selected
once the other values found an equilibrium, as any results other then this would
lower accuracy. Once the equilibrium was found this value would increase,
optimizing the speed of the test.     \item graph depth - the maximum depth the
graph algorithm would go before guessing.  As with important words, it was
thought that it would find the best compromise for accuracy initially, and as
the graph became more accurate, it would be increased to allow for faster
runtimes. \end{itemize}

The second group is the ``inflation limiting'' group.  This group consists of
values that can directly affect accuracy, and were added to the meta-variables
specifically to counter inflation of edge weight values over time.  The hopes
were this would combat dataset biases.

inflation limiting \begin{itemize}     \item Positive Outcome - This was the
value gained for each edge used in determining a correct word sense
disambiguation.      \item Negative Outcome - This was the value removed for
each edge used in determining an incorrect word sense disambiguation.
\end{itemize}

These groups, and the meta-learning component allowed for a great chance to add
accuracy in modelling the biological/cognitive aspect of memory and learning,
with a great chance of increasing overall accuracy and efficiency. Along with
this there was also a number of possible benefits such as a scaling reward and
punishment system, and an algorithm that was slower and more deliberate in its
searching when its accuracy was suspect, and faster when the learning component
was in the ``zone''.  It also had the added effect of eliminating the guess work
in initial assignment of these values.
