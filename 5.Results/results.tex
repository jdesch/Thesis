%%---------------Chapter 5------------------------------ 
\chapter{Results}

In this section the results of the human memory model will be presented.
Interpretation of the results and commentary on both what worked and what did
not work will be discussed. An overview of the approach, and the novelty of its
implementation will be looked at. Finally, The relative success of the work will
be outlined.

The comparisons made in the chapter are broken down into quantitative results
and discussion sections. The quantitative results section compares the current
systems performance to other systems that had competed in the Senseval-2
contest, including the baseline algorithm.  The discussion section generalizes
the trends the data from the quantitive section puts forth. This section
analyses pieces of data that seem interesting or inconsistent, hypothesising the
reasons why, and adding insight into potential issues.

\section{Quantitative results}

\begin{table}[htp]
	\begin{center}
		\begin{tabular}{|l|l|l|}
			\hline
				\multicolumn{3}{|c|}{\bf 5-fold cross validation results } \\
				\hline
				{\bf test instances} & {\bf accuracy} & {\bf std. dev.} \\ \hline 
				3045 & 0.5997 & 0.0284    \\ \hline 
				3044 & 0.5957 & 0.0279 \\ \hline 
				3045 & 0.5934 & 0.0287 \\ \hline 
				3045 & 0.6020 & 0.0286 \\ \hline 
				3045 & 0.5915 & 0.0284 \\ \hline
		\end{tabular}
		\caption{Results Table \label{table:RESULTS}}
	\end{center}
\end{table}

In table \ref{table:RESULTS} we see The results of each of the 5-fold cross
validation test instances run on the current system.  With an overall average of
59.7\% this test fell just within a standard deviation of the baseline result of
56.9\%. The baseline algorithm for the Senseval-2 contest performs surprisingly
well given it chooses the word sense with the highest seen likelihood at each
step.  As seen in table \ref{table:SENSEVAL2RESULTS}, it also outperformed a
majority of other systems. Added to the table (in bold) are the results from the
current learning method, and a self-created implementation of the baseline test.
The self-created implementation of the baseline test falls within 0.4\% of the
competitions baseline test, validating both the datasets, and the testing
approach used.

\begin{table}[htp]
	\centering 
	\begin{tabular}{|l|l|l|l|}
		\hline
		  \multicolumn{4}{|c|}{\bf English all words - fine-grained scoring } \\
		\hline
		{\bf precision } & {\bf recall} & {\bf attempted} & {\bf system} \\ \hline 
		0.690	&	0.690	&	100\%	&		SMUaw \\ \hline
		0.636	&	0.636	&	100\%	&		CNTS-Antwerp \\ \hline
		0.618	&	0.618	&	100\%	&		Sinequa-LIA - HMM \\ \hline
		
		{\bf 0.597} & {\bf 0.597} & {\bf 100\% } &	{\bf Current technique} \\ \hline	
		
		0.575	&	0.569	&	98.91\% &		UNED - AW-U2 \\ \hline
		
		{\bf 0.573} & {\bf 0.573\% } & {\bf 100\%} &	{\bf self-created baseline implementation } \\ \hline
		
		0.569	&	0.569	&	100\%		&	Baseline \\ \hline
		0.556	&	0.550	&	98.908\%	&	UNED - AW-U	 \\ \hline
		0.475	&	0.454	&	95.552\%	&	UCLA - gchao2 \\ \hline
		0.474	&	0.453	&	95.552\%	&	UCLA - gchao3 \\ \hline
		0.416	&	0.451	&	98.5\%		&	CL Research - DIMAP	 \\ \hline
		0.451	&	0.451	&	100\%		&	CL Research - DIMAP (R)	 \\ \hline
		0.500	&	0.449	&	89.729\%	&	UCLA - gchao \\ \hline
		0.360	&	0.360	&	99.96\%		&	Universiti Sains Malaysia 2	 \\ \hline
		0.748	&	0.357	&	47.756\%	&	IRST \\ \hline
		0.345	&	0.338	&	97.897\%	&	Universiti Sains Malaysia 1 \\ \hline
		0.336	&	0.336	&	99.96\%		&	Universiti Sains Malaysia 3 \\ \hline
		0.572	&	0.291	&	50.789\%	&	BCU - ehu-dlist-all \\ \hline
		0.440	&	0.200	&	45.37\%		&	Sheffield \\ \hline
		0.566	&	0.169	&	29.883\%	&	Sussex - sel-ospd \\ \hline
		0.545	&	0.169	&	31.055\%	&	Sussex - sel-ospd-ana \\ \hline
		0.598	&	0.140	&	23.332\%	&	Sussex - sel \\ \hline
		0.328	&	0.038	&	11.646\%	&	IIT 2 \\ \hline
		0.294	&	0.034	&	11.646\%	&	IIT 3 \\ \hline
		0.287	&	0.033	&	11.646\%	&	IIT 1 \\ \hline
	\end{tabular}
	\caption{Senseval 2 Results, current technique and self-baseline implementation added \label{table:SENSEVAL2RESULTS}}
\end{table}

The information in the regression analysis of training
accuracy (figure \ref{fig:OVERALLTRAININGACCURACY}) holds the results of all instances
over the entire 5-fold cross validation process, each dot represents an accuracy
calculation during each of the training runs. The x-axis represents the 
number of training instances processed up to that point (x1000), and the y-axis 
represents the accuracy. Each fold consists of approximately 12000 training 
instances, and after each fold, all previous learning data structures are reset, 
the only hold over being the changes to the meta-variables. The first trend is shown 
with the red line, this shows a steady improvement over the entire test. 
The 2.5\% gain on the average accuracy over the 5-folds can be contributed to the 
meta-learning component, and the preserved meta-variables mentioned earlier. The 
other important trend is found in looking at the general
improvement of values over one test fold, to make it more clear the regression
analysis of one fold has can be seen in figure \ref{fig:ONEFOLDACCURACY}.  The important
thing to note when looking at the training data set is that the statistical component was
pre-optimized during the training bootstrap phase.  Therefore since the
statistical component did not change over the life of the test, the improvement
can be attributed to the connective component, as it is still being optimized. 
While the meta-learning changes are cumulative over the whole graph, the nearly 
5\% increase in accuracy over the course of a single folds graph can be credited 
to the training of the connective component. Furthermore, as the accuracy 
increase over the entire test at a steady rate related to the meta-learning changes, and the rate of
improvement during a test fold is more accelerated, it can be seen that both
components independently improve accuracy.  From these assumptions we can then 
posit, accuracy improvements in one fold is improved in the majority by 
optimization in the connective component.  Overall accuracy improvement, or 
perhaps refinement over the course of retraining the learning components can be 
attributed to the meta-variables, and the training done by the meta-learning 
component. Furthermore, the results show conclusively that the accuracy and 
number of instances are positively correlated, therefore adding the meta-learning 
components and the connective components have added value.

%% http://nd.edu/~rwilliam/stats2/l26.pdf for serially correlation understanding

Table \ref{table:STATENTIRE} and \ref{table:STATFOLD} show statistics about the 
regression analysis tests on figure \ref{fig:OVERALLTRAININGACCURACY} and 
\ref{fig:ONEFOLDACCURACY}. The tables attempt to determine a formula for the 
accuracy given the number of instances. That formula is $y = \alpha + \beta x + \epsilon$,
in the case of figure \ref{fig:OVERALLTRAININGACCURACY}, we see that starting at 
a baseline of 60.5\%, each instance increases the accuracy by $(3.757\pm1.614)*10^{-4}$.
The confidence value of $\alpha$ shows that the confidence interval will 
incorporate 99\% of all instances results, where as $\beta$ confidence interval
incorporate 95\% of all instances results.  The t-statistic measures how 
reasonable the estimation is, the closer to zero, the more reasonable. The 
extremely large value in both figures $\alpha$ variable shows the values are 
serially correlated.  This makes sense as this value is highly dependent on its 
previous values, in this case estimation errors in earlier instance runs tend to 
be correlated with estimation errors in newer runs. This is expected when 
collecting data repeatedly across time, and will not affect the unbiasedness or 
consistency of estimates, it does usually give a more precise confidence interval 
than it actually is.  The $\beta$ variable t-statistic being greater then 2 means 
it is significant and has a high impact on the accuracy.  This is also expected, 
as the increase in instances being trained against should increase accuracy. The 
p-values of all the variables are less then the remainder confidence percentage, 
This indicates the results shown are non-random, and the variables have an 
effect on the dataset.

\begin{figure}[htp]
	\begin{center}
		\includegraphics[width=0.90\linewidth]{figures/charts/results_regression_analysis.pdf}
		\caption{regression analysis of training accuracy (x Instances x1000, y Accuracy) \label{fig:OVERALLTRAININGACCURACY}}
	\end{center}
\end{figure}

\begin{table}[htp]
	\begin{tabular}{|l|l|l|l|l|}
		\hline \multicolumn{5}{|c|}{\bf regression analysis statistics (entire training) } \\ \hline
			{\bf model} & \multicolumn{4}{|c|}{$y = \alpha + \beta x + \epsilon$ where y is accuracy and x is instances} \\ \hline
		            & {\bf estimate} & {\bf confidence} & {\bf t-statistic} & {\bf p-value } \\ \hline
		{\bf $\alpha$}    & $0.6058\pm0.0047$   & $>99\%$            & 128         & $7.3*10^{-4}$ \\ \hline
		{\bf $\beta$}     & $(3.757\pm1.614)*10^{-4}$ & $>95\%$        & 2.3         & 0.024 \\ \hline
		\end{tabular}
		\caption{statistical values of regression analysis on entire training data}
		\label{table:STATENTIRE}       
\end{table}

\begin{figure}[htp]
	\begin{center}
		\includegraphics[width=0.90\linewidth]{figures/charts/one_fold_regression.pdf}
		\caption{regression analysis of training accuracy over one fold (x Instances, y Accuracy) \label{fig:ONEFOLDACCURACY}}
	\end{center}
\end{figure}

\begin{table}[htp]
	\centering 
	\begin{tabular}{|l|l|l|l|l|}
		\hline \multicolumn{5}{|c|}{\bf regression analysis statistics (one training fold) } \\ \hline
			{\bf model} & \multicolumn{4}{|c|}{$y = \alpha + \beta x + \epsilon$  where y is accuracy and x is instances} \\ \hline
		            & {\bf estimate} & {\bf confidence} & {\bf t-statistic} & {\bf p-value } \\ \hline
		{\bf $\alpha$}    & $0.5895\pm0.0091$   & $>99\%$            & 65         & $9.8*10^{-24}$ \\ \hline
		{\bf $\beta$}     & $(3.498\pm1.428)*10^{-6}$ & $>95\%$        & 2.4         & 0.024 \\ \hline
		\end{tabular}
		\caption{statistical values of regression analysis over one fold \label{table:STATFOLD}}
\end{table}

To validate how well the memory model setup worked, we will again look to the
training data leading up to the above test results.  In the training results
table \ref{table:TRAININGRESULTS} we see the total results for all of the
training of the cross-validation runs.  In total 60700 training instances were
run, of which 37587 were successful.  The statistical component, which represents 
the snap judgement of unconscious mind was correct
59.6\% of the time, which made up 96.2\% of the correct results.  The
connective component, which represented the deductive mind was correct 37.3\% 
of the time, and made up 62.5\% of the
correct results.  Although not a completely fair comparison, as the statistical
component was pre-trained during the test bootstrap phase, and the connective
component was being trained throughout the training run.  It is obvious that
there is a large overlap between the accuracy of the two components, however
even from table \ref{table:TRAININGRESULTS} we can make out that the statistical
component does not dominate the graph component.

\begin{table}[htp]
	\centering 
	\begin{tabular}{|l|l|l|l|l|}
		\hline
			\multicolumn{5}{|c|}{\bf training results } \\
		\hline
		{\bf total inst.} & {\bf correct } & {\bf percentage} & {\bf connect. correct } & {\bf stats correct } \\ \hline 
		60700				  & 37587		   & 61.92\%		  & 22618				  & 36166    \\ \hline 
	\end{tabular}
	\caption{overview of training results. \label{table:TRAININGRESULTS}}
\end{table}

In table \ref{table:ALLACCURACY} the accuracy section is the outcome of the
algorithm used to determine disambiguation. The statistical and connective
accuracy is how often the individual components chose the correct
disambiguation.  The exclusive statistic and connective is how often the
components are correct while the other component was incorrect.  Finally, the
optimal accuracy is the outcome that could be attained provided the meta-
learning component performed optimally and chose the correct value from the
components every time a correct value was presented. This table and the
information presented earlier clearly show three things: 
\begin{itemize}
	\item Both the statistical and connective components can add value to the 
	outcome. That is to say, the two components can come to a higher accuracy 
	together then they could by themselves.     
	\item The meta-learning component is unfairly biased towards the statistical
	component.     
	\item There is something drastically increasing the accuracy of the connective component in the training results. 
\end{itemize}

Clearly in the training data, and somewhat in the test data we see a benefit to
the two input systems. Although the results are slightly higher then that of the
baseline test, there is potential for significant improvement.  The training
results point to a deficiency in the way the meta-learning component weighs and
chooses the target disambiguation.  With the connective component improving its
accuracy beyond the statistical component, the training accuracy should have a
higher rate of selection. We see in table \ref{table:BIASNESS} that this is not
the case. This leads to the second point, the meta-learning component may have a
more than human bias.  In the most extreme case, it chose to go with the
statistical component 99.01\% of the time despite the connective component
exclusively selecting the correct value 12.12\% of the time, and having an
accuracy of 39.44\%. Finally, the large difference in the connective accuracy
between the training and testing results is due to the differences in learning
behaviour.  The training results allow the connective data and the meta-learning
component to modify its learning behaviour per instance, where as the testing
results have both of these locked in.  Quicker feedback results in better
accuracy, and was done to simulate human testing feedback. Where as to follow
the rules of the Senseval 2 test, this mechanism was not used during the testing
data.  This trait appears to positively effect the connective component's
accuracy, but has very little effect on the meta-learning component.

\begin{table}[htp]
    \centering 
	\begin{tabular}{|l|l|l|l|l|l|}
		\hline
		\multicolumn{6}{|c|}{\bf training results } \\ \hline
		{\bf acc.} & {\bf stat. acc.} & {\bf excl. stat.}& {\bf connect. acc. } & {\bf excl. acc. } & {\bf opt. acc.} \\ \hline
		60.84\%	   &  59.54\%    & 26.20\%   &  55.02\%  & 21.67\%   & 81.22\% \\ \hline
		60.30\%    &  59.62\%    & 26.11\%   &  57.45\%  & 23.94\%   & 83.56\% \\ \hline
		60.90\%    &  59.68\%    & 24.54\%   &  59.51\%  & 24.37\%   & 84.05\% \\ \hline
		60.60\%    &  59.45\%    & 25.06\%   &  60.23\%  & 25.88\%   & 85.33\% \\ \hline
		60.75\%    &  59.72\%    & 21.33\%   &  65.02\%  & 26.63\%   & 86.34\% \\ \hline
		\multicolumn{6}{|c|}{\bf testing results } \\ \hline
		{\bf acc.} & {\bf stat. acc.} & {\bf excl. stat.} & {\bf connect. acc. } & {\bf excl. acc. } & {\bf opt. acc.} \\ \hline
		59.90\%    &  59.77\%   & 31.43\%   &  39.34\%   & 11.00\%   & 70.77\% \\ \hline
		59.58\%    &  59.55\%   & 34.83\%   &  36.05\%   & 11.34\%   & 70.88\% \\ \hline
		59.30\%    &  59.30\%   & 32.26\%   &  38.70\%   & 11.66\%   & 70.96\% \\ \hline
		60.20\%    &  60.23\%   & 32.91\%   &  39.44\%   & 12.12\%   & 72.35\% \\ \hline
		59.15\%    &  59.15\%   & 27.75\%   &  43.58\%   & 12.18\%   & 71.33\% \\ \hline
	\end{tabular}
	\caption{various accuracy comparisons between components \label{table:ALLACCURACY}}
\end{table}

\begin{table}[htp]
	\centering 
	\begin{tabular}{|l|l|l|l|}
		\hline
		\multicolumn{4}{|c|}{\bf training results } \\ \hline
		{\bf stat. chosen } & {\bf conn. chosen } & {\bf stat acc.} & {\bf conn. acc.}\\ \hline
		94.23\%             & 5.77\%              &  59.54\%        & 55.02\% \\ \hline
		93.53\%             & 6.47\%              &  59.62\%        & 57.45\% \\ \hline
		94.26\%             & 5.74\%              &  59.68\%        & 59.51\% \\ \hline
		94.43\%             & 5.57\%              &  59.45\%        & 60.23\% \\ \hline
		93.01\%             & 6.99\%              &  59.72\%        & 65.02\% \\ \hline
	    \multicolumn{4}{|c|}{\bf training results } \\ \hline
		{\bf stat. chosen } & {\bf conn. chosen } & {\bf stat acc.} & {\bf conn. acc.}\\ \hline
		97.67\%             & 2.33\%              &  59.77\%        & 39.34\% \\ \hline
		98.78\%             & 1.22\%              &  59.55\%        & 36.05\% \\ \hline
		98.49\%             & 1.51\%              &  59.30\%        & 38.70\% \\ \hline
		99.01\%             & 0.99\%              &  60.23\%        & 39.44\% \\ \hline
		98.85\%             & 1.14\%              &  59.15\%        & 43.58\% \\ \hline
	\end{tabular}
	\caption{component biasness \label{table:BIASNESS}}
\end{table}

\section{Discussion}

The results are competitive with other systems from the Senseval 2 test, it
performs respectfully if a little low given the 10 years that have passed.
However the novel approach of applying a human-like memory model to this test
has given some insight into its potential as a learning algorithm, the
challenges this method faces, and some of the understanding required to improve
its likeness to human memory and overall accuracy.

The outcome and results show a successful interpretation of the human memory
models discussed in earlier chapters. In table \ref{table:ALLACCURACY} we see in
the training results the connective component grow from 55\% to 65\% accuracy
over the course of the five training runs. The test results are not as extreme
but a steady increase in accuracy is seen.  The ability to use a guess or gut
feeling (the statistical component) to have a decent chance at being correct,
while learning some more of the deeper connections over time between words used
and disambiguations is very human like.  When required to give an answer, the
system used the most seen candidate, while the more complex cases, and patterns
were being learned by the ``long term'' memory component.  Although the meta-
learning component did not allow the two components to be completely successful,
as it was overly biased and flawed in its decision making, it did do some things
well. It improved or better optimized the systems ability to learn, by meta-
optimizing the learning parameters. Like a human, repeated problem solving
approaches will not only improve the area being solved, it will also sharpen the
problem solving toolset. This optimization system had the meta-learning
component make hypothesis on what would work well, test its hypothesis over a
training run, and validate whether the change it proposed had a positive effect
on the system. This allowed for small incremental gains to carry over to
successive runs, allowing each initial state to spend less time bootstrapping
and more time increasing accuracy. On the other hand, the meta-learning
components algorithm used to evaluate which of the two components to use had
faults.  In looking at the designed algorithm, and the outcome of its training
and testing, it is likely the perfect memory of the algorithm caused issues.
The initial state of the statistical component having a decent accuracy and the
necessity for the connective component to bootstrap itself before having a
comparative accuracy allowed the algorithm to put too much weight on the
statistic accuracy.  This bias is useful initially, as it keeps the accuracy
high, but fails to adjust over the thousands of iterations.

In this work the memory system uses three components in an attempt to simulate
not only the different components of the brain, but the multiple senses humans
use to better store information. While the five human senses were abstracted
down to two components with different views on the same data, these components
had a dual-purpose of also simulating the different types of memory.  The
increased accuracy in using differing views on the data provided greater overall
accuracy. Although simulating additional senses from the data provided by the
Senseval 2 datasets would be difficult, the benefits of the additional
information would likely both increase accuracy, and better simulate human
memory.

